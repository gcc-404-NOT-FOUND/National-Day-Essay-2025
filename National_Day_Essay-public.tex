% __        _________________________________     ___
%|  |      |   ___   |___    ___|   _____|\  \   /  /
%|  |      |  |___|  |   |  |   |  |_____  \  \_/  /
%|  |      |   ___   |   |  |   |   _____|  |  _  |
%|  |______|  |   |  |   |  |   |  |_____  /  / \  \
%|_________|__|   |__|   |__|   |________|/__/   \__\

%序言
\documentclass[UTF8,12pt,twoside]{article}
\usepackage{ctex}
\usepackage[top=3.2cm,bottom=3.2cm,left=2.1cm,right=2.1cm,a4paper]{geometry}
\usepackage{indentfirst} %首段缩进
\usepackage{fancyhdr} %页眉页脚
\usepackage{setspace} %行间距
\usepackage{xcolor} %字体颜色
\usepackage{eso-pic} %水印
\usepackage{graphicx} %加载图片
\usepackage{hyperref} %超链接

\doublespacing %1.5倍行距
\setlength{\parindent}{2em} %首段缩进2格

%正文
\begin{document}

%标题
\begin{spacing}{0} %此处使用1倍行距
    \begin{center}
        \LARGE{\textbf{\textsf{传承奋斗之薪火,担当时代之责任}}}
    \end{center}
    \begin{flushright}
        \large{\textsf{——庆祝中华人民共和国成立76周年}}
    \end{flushright}
\end{spacing}
\vspace{3em}

%页眉页脚
\pagestyle{fancy}
\fancyhf{} %清空默认设置
%   页眉
\fancyhead[C]{\LaTeX\\侵权立改}
%   页脚
\fancyfoot[C]{\scriptsize{源代码:\textcolor{blue}{\url{https://github.com/gcc-404-NOT-FOUND/National-Day-Essay-2025}}}}
\fancyfoot[LE]{\thepage}
\fancyfoot[RO]{\thepage}

%水印
\AddToShipoutPictureBG{
    \AtPageLowerLeft{
        \rotatebox{60}{\makebox[\paperwidth]{\textcolor{gray!18}{\scalebox{4}{\Huge{Administrator}}}}}
    }
}

%正文
%   第一段
从1921年7月23日中国共产党成立,到1949年10月1日中华人民共和国开国大典,
再到今天新中国成立76周年,中国共产党奋斗了104年,历经
新民主主义革命时期、
社会主义革命和建设时期、
改革开放和社会主义现代化建设新时期和
中国特色社会主义新时代\footnote{时间分别为:1921年-1949年,1949年-1978年,1978年-2012年,2012年-。}。
如今,我们作为新时代青少年,应当传承社会主义建设先辈之薪火,担当时代之责任。\par %中心论点
\vspace{1em}
%   第二段
以自尊立骨,承先烈风骨,筑牢担当之基。 %分论点 1
自尊是对自我价值的肯定,包括自我尊重和赢得他人的尊重。
古人云:“人必自重而后人重之。”
一个不懂得尊重自身历史与根源的人,无法获得他人的尊重,也无法真正肩负起未来的责任。
人民解放军之所以能杀出一条血路,是因为他们对脚下土地和身后人民深沉的自尊,绝不容许国家与民族被轻贱。
所以,做到自尊,先学会维护自己的人格尊严,学会不断提升自己,学会尊重他人。\par
\vspace{1em}
%   第三段
以自信扬帆,继开创伟业,把稳担当之舵。 %分论点 2
自信就是相信自己,是一个人对自身能力的肯定,是一个人自身能力的彰显,是一个人精神风貌的体现。
从北斗导航系统全面建成,到“嫦娥”探月、“天问“探火,这些科技成就的背后,是几代航天人接续奋斗的结果。
站在先辈的肩膀上,作为青少年,我们要找准定位,扬长避短;
专注当下,积极行动;
勇于探索,增强底气。
我们要怀着对未来的憧憬,坚定民族自信心,在全面建设社会主义现代化国家的新征程上做自信的中国人。\par
\vspace{1em}
%   第四段
以自强致远,续奋斗征程,挥洒担当之汗。 %分论点 3
自强是自我勉励、发愤图强。自强是一种奋发进取的精神状态,是一种不断完善自我、超越自我的人生追求。
在脱贫攻坚战场上,
无数像黄文秀那样的“80后”“90后”乃至“00后”驻村干部用脚步丈量土地,用智慧点燃希望,甚至献出生命;
在科研攻关一线,年轻的科研团队挑大梁、担重任,如DeepSeek的开发团队;
在抗疫斗争中,当别人躲在家里等待防疫物资的送达时,“大白”
\footnote{指新冠疫情中身穿防护服的医护人员,由防护服的颜色为白色而得名。}
逆行出征……
他们用行动证明,新时代青年正在各自的赛道上自强奋斗,将先辈的薪火化为照亮前路的光芒。
对于我们来说,我们要树立远大理想;求得真学问,练就真本领;在苦干实干中磨砺成长,砥砺前行。\par
\vspace{1em}
%   第五段
综上所述,自尊让我们铭记来路,珍视先辈用尊严铸就的今天;
自信让我们认清道路,坚信先辈用实践验证的方向;
自强则让我们开拓前路,用先辈般的奋斗去创造新的辉煌。
唯有如此,新时代青少年才能真正做到“传承奋斗之薪火,担当时代之责任”,
在历史的星河中,留下属于我们这一代人的璀璨光芒。

%后记
\begingroup
    \renewcommand{\thefootnote}{} %临时去编号
    \footnotetext{引用内容:义务教育教科书人教版七年级下册《道德与法治》}
\endgroup

\end{document}